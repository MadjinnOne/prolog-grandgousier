% =========================
% SLIDES.TEX (compile OK)
% =========================

\section{Objectif}
\begin{frame}{Objectif}
\begin{itemize}
  \item Bot conversationnel en \textsc{Prolog} pour conseiller des vins.
  \item Deux fonctions principales :
  \begin{itemize}
    \item recommander (appellation, prix, plats) ;
    \item expliquer (nez, bouche, description, appellation).
  \end{itemize}
  \item Approche symbolique : mots-cles ponderes + regles + motifs.
\end{itemize}
\end{frame}

\section{Architecture}
\begin{frame}{Architecture generale}
\begin{itemize}
  \item Deux modules :
  \begin{enumerate}
    \item base de connaissances (faits) ;
    \item moteur de dialogue (normalisation, regles, reponse).
  \end{enumerate}
  \item Chaine :
  \texttt{read\_atomics/1} $\rightarrow$ \texttt{produire\_reponse/2} $\rightarrow$ \texttt{ecrire\_reponse/1}.
\end{itemize}
\end{frame}

\section{Base de connaissances}
\begin{frame}[fragile]{Base de connaissances : representation}
\begin{itemize}
  \item Identifiant stable par vin (atom Prolog).
  \item Faits : \texttt{nom/2}, \texttt{prix/2}, \texttt{provenance/2}, \texttt{appellation/2}.
  \item Textes : \texttt{nez/2}, \texttt{bouche/2}, \texttt{description/2} (listes de lignes).
\end{itemize}

\vspace{0.15cm}
\begin{lstlisting}[style=prologstyle]
nom(chambolle_musigny_premier_cru_2012,
 'Chambolle Musigny 1er Cru 2012 - Les Noirots').
prix(chambolle_musigny_premier_cru_2012, 63.85).
appellation(chambolle_musigny_premier_cru_2012, chambolle_musigny).
provenance(chambolle_musigny_premier_cru_2012, bourgogne).
\end{lstlisting}
\end{frame}

\section{Normalisation}
\begin{frame}[fragile]{Normalisation de l'entree}
\begin{itemize}
  \item Objectif : rendre le matching robuste aux variantes d'ecriture.
  \item Etapes :
  \begin{itemize}
    \item unifier les noms de vins (long/court/compact) ;
    \item normaliser les appellations et les plats ;
    \item decomposer certains tokens composes.
  \end{itemize}
\end{itemize}

\vspace{0.15cm}
\begin{lstlisting}[style=prologstyle]
normaliser_question(Lin,Lout) :-
  nom_vins_uniforme(Lin,L1),
  normaliser_appellations_tokens(L1,L2),
  normaliser_plats_tokens(L2,L3),
  expand_compound_tokens(L3,L4),
  maplist(normaliser_mot,L4,Lout).
\end{lstlisting}
\end{frame}

\section{Moteur de regles}
\begin{frame}[fragile]{Mots-cles ponderes et regles}
\begin{itemize}
  \item \texttt{mclef/2} priorise l'intention (nez/bouche > vin/vins).
  \item Une regle : \texttt{regle\_rep(MotCle, Id, Pattern, Rep) :- Conditions.}
  \item Matching : sous-liste dans la question normalisee.
\end{itemize}

\vspace{0.15cm}
\begin{lstlisting}[style=prologstyle]
mclef(nez,10).
mclef(bouche,10).
mclef(appellation,8).
mclef(vins,5).

regle_rep(nez,1,[quel,nez,presente,le,Vin],Rep) :-
  nez(Vin,Rep), memoriser_vin(Vin).
\end{lstlisting}
\end{frame}

\section{Contexte}
\begin{frame}[fragile]{Contexte minimal : dernier vin}
\begin{itemize}
  \item Gere les questions elliptiques (ex. "puis-je le boire maintenant").
  \item Memoire : \texttt{dernier\_vin/1} mis a jour quand un vin est cite.
\end{itemize}

\vspace{0.15cm}
\begin{lstlisting}[style=prologstyle]
:- dynamic dernier_vin/1.

memoriser_vin(Vin) :-
  retractall(dernier_vin(_)),
  asserta(dernier_vin(Vin)).
\end{lstlisting}
\end{frame}

\section{Tests}
\begin{frame}[fragile]{Validation : tests automatises (plunit)}
\begin{itemize}
  \item Suite \texttt{plunit} :
  \begin{itemize}
    \item base (provenance/appellation pour chaque vin) ;
    \item normalisation (formes compactes, variantes) ;
    \item reponses (Bourgogne, prix, nez/bouche, plats, cas vides).
  \end{itemize}
\end{itemize}

\vspace{0.15cm}
\begin{lstlisting}[style=prologstyle]
test(all_wines_have_taxonomy) :-
  findall(V, nom(V,_), Vins),
  forall(member(V,Vins),
         (provenance(V,_), appellation(V,_))).
\end{lstlisting}
\end{frame}

\section{Limites}
\begin{frame}{Limites et perspectives}
\begin{itemize}
  \item Limites :
  \begin{itemize}
    \item couverture linguistique dependante des patterns ;
    \item ambiguite si plusieurs mots-cles forts ;
    \item heuristiques sensibles au lexique.
  \end{itemize}
  \item Pistes :
  \begin{itemize}
    \item enrichir synonymes et patrons ;
    \item memoriser aussi le dernier filtre (appellation/prix) ;
    \item scoring global avant choix final.
  \end{itemize}
\end{itemize}
\end{frame}

\section{Conclusion}
\begin{frame}{Conclusion}
\begin{itemize}
  \item Base riche + moteur de regles robuste.
  \item Normalisation : comprehension fiable sans NLP complexe.
  \item Tests \texttt{plunit} : fiabilite et non-regression.
\end{itemize}
\centering\vspace{0.4cm}
\textbf{Merci !}
\end{frame}
