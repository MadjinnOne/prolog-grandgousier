\section{Patrons de formulation (\texttt{regle\_rep/4})}
\label{ann:patterns}
Cette section recense de manière exhaustive les patrons (listes de tokens) reconnus par le bot au niveau des règles \texttt{regle\_rep/4}, regroupés par type de question.
Les variables suivantes désignent :
\begin{itemize}
  \item \texttt{Vin} : un vin (nom d'une entrée de la base),
  \item \texttt{App} : une appellation,
  \item \texttt{Plat} : un plat (ou ingrédient),
  \item \texttt{X, Y, Min, Max} : des nombres (prix),
  \item \texttt{Det} : un déterminant/segment optionnel (ex. \texttt{le}, \texttt{ce}, etc.).
\end{itemize}


\footnotesize
\begin{multicols}{2}

\subsection{Bouche d'un vin}
\begin{verbatim}
[que, donne, le, Vin, en, bouche]
[que, donne, Vin, en, bouche]
[comment, est, Vin, en, bouche]
[que, donne, Vin, en, bouche, '?']
[bouche, de, Vin]
\end{verbatim}

\subsection{Nez d'un vin}
\begin{verbatim}
[quel, nez, presente, le, Vin]
[quel, nez, presente, Vin]
[quel, nez, pour, Vin]
[quel, nez, pour, Vin, '?']
[nez, de, Vin]
\end{verbatim}

\subsection{Description d'un vin /  appellation}
\begin{verbatim}
[pourriezvous, men, dire, plus, sur, le, Vin]
[pourriezvous, men, dire, plus, sur, Vin]
[pourriez, men, dire, plus, sur, le, Vin]
[pourriez, men, dire, plus, sur, Vin]
[que, pouvezvous, me, dire, sur, le, Vin]
[parlez, moi, du, Vin]
[parlez, moi, de, Vin]
[parle, moi, du, Vin]
[parlezvous, du, Vin]
\end{verbatim}

\subsection{Recommandations par appellation}
\begin{verbatim}
[quels, vins, de, App, me, conseillezvous]
[quels, vins, de, App, me, conseillez, vous]
[auriezvous, des, vins, de, App]
[avezvous, des, vins, de, App]
[quel, vin, de, App, me, conseillezvous]
[quel, vin, de, App, me, conseillez, vous]
[vous, auriez, un, App]
[vous, auriez, des, App]
[auriez, vous, un, App]
[auriez, vous, des, App]
\end{verbatim}

\subsection{Autres vins d'une appellation}
\begin{verbatim}
[auriezvous, dautres, vins, de, App]
[auriezvous, d, autres, vins, de, App]
[auriez, vous, d, autres, vins, de, App]
[auriezvous, autres, vins, de, App]
[auriez, vous, autres, vins, de, App]
\end{verbatim}

\subsection{Accessibilité / ''prêt à boire''}
\begin{verbatim}
[est, ce, un, vin, que, je, peux, boire, tout, de, 
suite, Vin]
[est, ce, un, vin, que, je, peux, boire, tout, de, suite, 
Det, Vin]
[est, ce, que, je, peux, boire, Vin, tout, de, suite]
[est, ce, que, je, peux, boire, Det, Vin, tout, de, suite]
[est, ce, que, je, peux, boire, Vin, maintenant]
[est, ce, que, je, peux, boire, Det, Vin, maintenant]
[puis, je, boire, Vin, maintenant]
[puis, je, boire, Det, Vin, maintenant]
[puis, je, deja, boire, Vin]
[puis, je, deja, boire, Det, Vin]
[je, peux, deja, boire, Vin]
[je, peux, deja, boire, Det, Vin]
[peut, on, boire, Vin, tout, de, suite]
[peut, on, boire, Det, Vin, tout, de, suite]
[peut, on, boire, Vin, maintenant]
[peut, on, boire, Det, Vin, maintenant]
[Vin, est, il, pret, a, boire]
[Det, Vin, est, il, pret, a, boire]
[je, peux, boire, Vin, tout, de, suite]
[je, peux, boire, Det, Vin, tout, de, suite]
[je, peux, boire, Vin, maintenant]
[je, peux, boire, Det, Vin, maintenant]
[est, ce, un, vin, que, je, peux, boire, tout, de, suite]
[est, ce, un, vin, que, je, peux, boire, maintenant]
[est, ce, que, je, peux, le, boire, tout, de, suite]
[est, ce, que, je, peux, le, boire, maintenant]
\end{verbatim}

\subsection{Filtrage par prix}
\begin{verbatim}
[auriezvous, des, vins, entre, X, et, Y, eur]
[auriez, vous, des, vins, entre, X, et, Y, eur]
[avezvous, des, vins, entre, X, et, Y, euros]
[avez, vous, des, vins, entre, X, et, Y, euros]
[auriezvous, des, vins, entre, X, et, Y]
[auriez, vous, des, vins, entre, X, et, Y, euros]
[auriezvous, des, vins, entre, X, et, Y, euros]
[avezvous, des, vins, a, moins, de, Max, euros]
[avezvous, des, vins, a, moins, de, Max]
[avezvous, des, vins, a, plus, de, Min, euros]
[avezvous, des, vins, a, plus, de, Min]
[des, vins, a, moins, de, Max, euros]
[des, vins, a, moins, de, Max]
[des, vins, a, plus, de, Min, euros]
[des, vins, a, plus, de, Min, eur]
[des, vins, a, plus, de, Min]
\end{verbatim}

\end{multicols}

\subsection{Accords mets--vins}
\begin{verbatim}
[je, cuisine, du, Plat, quel, vin, me, conseillezvous]
[je, cuisine, des, Plat, quel, vin, me, conseillezvous]
[Plat, quel, vin, me, conseillezvous]
[je, fais, un, Plat, quel, vin, servir]
[quel, vin, avec, du, Plat]
[quel, vin, avec, des, Plat]
[quel, vin, pour, Plat]
\end{verbatim}

\subsection{Définition d'une appellation}
\begin{verbatim}
[que, recouvre, appellation, App]
[que, recouvre, appellation, App, '?']
[que, recouvre, l, ''', appellation, App]
[que, recouvre, l, ''', appellation, App, '?']
\end{verbatim}

\subsection{Fin de session}
\begin{verbatim}
[fin]
\end{verbatim}


\normalsize
\newpage
\section{Flux d’exécution}
\label{ann:produire}
\begin{footnotesize}
    
    

\begin{verbatim}
grandgousier/0
  ↓
lire_question/1
  ↓
normaliser_question/2
  ↓
produire_reponse/2
     ├─ (A) Préparation / repérage
     │     ├─ normaliser_question/2
     │     └─ mclef/2 + member/2
     │          → repère un “mot-clé” présent dans la question normalisée
     │
     ├─ (B) Sélection d’une règle de dialogue
     │     ├─ clause/2 sur regle_rep/4
     │     │     → récupère une règle candidate associée au mot-clé détecté
     │     └─ match_pattern/2
     │           → vérifie que le patron de mots attendu est bien présent
     │
     ├─ (C) Exécution de l’action (le vrai contenu de la réponse)
     │     └─ call(Body)
     │           → déclenche UN des “constructeurs” de réponse suivants (selon la règle)
     │
     │        1) Réponses de dégustation (vin explicite)
     │           ├─ bouche_reponse/2
     │           └─ nez_reponse/2
     │              ↳ s’appuient sur base_vins : bouche/2, nez/2 (+ mise en forme)
     │              ↳ appellent souvent memoriser_vin/1 pour le contexte
     │
     │        2) Réponse « parle / parlez-moi de … »
     │           └─ description_ou_appellation/2
     │              ↳ si on reconnaît un vin : description/2
     │              ↳ sinon : bascule vers definition_appellation/2
     │
     │        3) Recommandations / listes de vins
     │           ├─ reponse_conseil/3
     │           │    ↳ récupère des contraintes (appellation, etc.)
     │           │    ↳ filtre/sélectionne des vins via base_vins
     │           │    ↳ limite le nombre de résultats
     │           └─ repondre_vins_prix/...
     │                ↳ cas typiques « moins / plus ... euros »
     │
     │        4) Accords mets–vins (plats)
     │           └─ reponse_plat/2
     │              ↳ construit un profil de plat
     │              ↳ transforme en recommandations
     │              ↳ formate la sortie
     │
     │        5) « Puis-je le boire maintenant ? »
     │           └─ reponse_accessibilite/2
     │              ↳ utilise dernier_vin/1 si nécessaire
     │              ↳ applique une heuristique de scoring
     │
     └─ (D) Cas d’échec (fallback)
           → aucune règle ne correspond
  ↓
ecrire_reponse/1
\end{verbatim}

\end{footnotesize}


